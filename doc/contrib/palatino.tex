@c palatino.tex -- TeXinfo macros to set the default Roman font to Palatino
@c
@def@palatinoversion{2003-04-01.00}
@c
@c Copyright (C) 2003  Free Software Foundation, Inc.
@c
@c This palatino.tex file is free software; you can redistribute it and/or
@c modify it under the terms of the GNU General Public License as
@c published by the Free Software Foundation; either version 2, or (at
@c your option) any later version.
@c
@c This palatino.tex file is distributed in the hope that it will be
@c useful, but WITHOUT ANY WARRANTY; without even the implied warranty
@c of MERCHANTABILITY or FITNESS FOR A PARTICULAR PURPOSE.  See the GNU
@c General Public License for more details.
@c
@c You should have received a copy of the GNU General Public License
@c along with this texinfo.tex file; see the file COPYING.  If not, write
@c to the Free Software Foundation, Inc., 59 Temple Place - Suite 330,
@c Boston, MA 02111-1307, USA.
@c
@c You should use this module, go to your root .texi file and make sure
@c it reads like this:
@c     \input texinfo  @c -*-texinfo-*-
@c     @input palatino

@message{Loading palatino [version @palatinoversion]:}

@c Turn on the normal TeX command characters.
@catcode`\=0
\catcode`\%=14
% Set the font macro #1 to the font named #2#3. #4 is the point size.
% We save \setfont as \setfontorig, so we can restore it at the end of this
% section.
\catcode`\#=6
\let\setfontorig=\setfont
\def\setfont#1#2#3#4{\font#1=#2#3 at #4}
\catcode`\#=\other

% Set Palatino as the default roman font face
\def\rmfontprefix{ppl}

% Only define roman font attributes here.
\def\rmshape{r}
\def\rmbshape{b}
\def\bfshape{b}
\def\bxshape{b}
\def\itshape{ri}
\def\itbshape{bi}
\def\slshape{ro}
\def\slbshape{bo}
\def\scshape{rc}
\def\scbshape{bc}

\ifx\bigger\relax
  % not really supported.
  \def\mainmagstep{12pt}
  \setfont\textrm\rmfontprefix\rmshape{\mainmagstep}
\else
  \def\mainmagstep{10pt}
  \setfont\textrm\rmfontprefix\rmshape{\mainmagstep}
\fi
% Instead of cmb10, you many want to use cmbx10.
% cmbx10 is a prettier font on its own, but cmb10
% looks better when embedded in a line with cmr10.
\setfont\textbf\rmfontprefix\bfshape{\mainmagstep}
\setfont\textit\rmfontprefix\itshape{\mainmagstep}
\setfont\textsl\rmfontprefix\slshape{\mainmagstep}
\setfont\textsc\rmfontprefix\scshape{\mainmagstep}
\font\texti=zppler7m at \mainmagstep
\font\textsy=zppler7y at \mainmagstep

% A few fonts for \defun, etc.
\setfont\defbf\rmfontprefix\bxshape{10pt} %was 1314
\def\df{\let\tentt=\deftt \let\tenbf = \defbf \bf}

% Fonts for indices, footnotes, small examples (9pt).
\setfont\smallrm\rmfontprefix\rmshape{9pt}
\setfont\smallbf\rmfontprefix\bfshape{9pt}
\setfont\smallit\rmfontprefix\itshape{9pt}
\setfont\smallsl\rmfontprefix\slshape{9pt}
\setfont\smallsc\rmfontprefix\scshape{9pt}
\font\smalli=zppler7m at 9pt
\font\smallsy=zppler7y at 9pt

% Fonts for small examples (8pt).
\setfont\smallerrm\rmfontprefix\rmshape{8pt}
\setfont\smallerbf\rmfontprefix\bfshape{8pt}
\setfont\smallerit\rmfontprefix\itshape{8pt}
\setfont\smallersl\rmfontprefix\slshape{8pt}
\setfont\smallersc\rmfontprefix\scshape{8pt}
\font\smalleri=zppler7m at 8pt
\font\smallersy=zppler7y at 8pt

% Fonts for title page:
\setfont\titlerm\rmfontprefix\rmbshape{18pt}
\setfont\titleit\rmfontprefix\itbshape{18pt}
\setfont\titlesl\rmfontprefix\slbshape{18pt}
\let\titlebf=\titlerm
\setfont\titlesc\rmfontprefix\scbshape{18pt}
\font\titlei=zppler7m at 18pt
\font\titlesy=zppler7y at 18pt
\def\authorrm{\secrm}
\def\authortt{\sectt}

% Chapter (and unnumbered) fonts (17.28pt).
\setfont\chaprm\rmfontprefix\rmbshape{17.28pt}
\setfont\chapit\rmfontprefix\itbshape{17.28pt}
\setfont\chapsl\rmfontprefix\slbshape{17.28pt}
\let\chapbf=\chaprm
\setfont\chapsc\rmfontprefix\scbshape{17.28pt}
\font\chapi=zppler7m at 17.28pt
\font\chapsy=zppler7y at 17.28pt

% Section fonts (14.4pt).
\setfont\secrm\rmfontprefix\rmbshape{14.4pt}
\setfont\secit\rmfontprefix\itbshape{14.4pt}
\setfont\secsl\rmfontprefix\slbshape{14.4pt}
\let\secbf\secrm
\setfont\secsc\rmfontprefix\scbshape{14.4pt}
\font\seci=zppler7m at 14.4pt
\font\secsy=zppler7y at 14.4pt

% Subsection fonts (13.15pt).
\setfont\ssecrm\rmfontprefix\rmbshape{13.15pt}
\setfont\ssecit\rmfontprefix\itbshape{13.15pt}
\setfont\ssecsl\rmfontprefix\slbshape{13.15pt}
\let\ssecbf\ssecrm
\setfont\ssecsc\rmfontprefix\scbshape{13.15pt}
\font\sseci=zppler7m at 13.15pt
\font\ssecsy=zppler7y at 13.15pt
% The smallcaps and symbol fonts should actually be scaled \magstep1.5,
% but that is not a standard magnification.

% Fonts for short table of contents.
\setfont\shortcontrm\rmfontprefix\rmshape{12pt}
\setfont\shortcontbf\rmfontprefix\bxshape{12pt}
\setfont\shortcontsl\rmfontprefix\slshape{12pt}

% Set keyfont as well.
\setfont\keyrm\rmfontprefix\rmshape{8pt}
\font\keysy=zppler7y at 9pt

\let\setfont=\setfontorig
\def\setfontorig{\relax}

% Restore the TeXinfo character set.
\catcode`\\=\active
@catcode`@%=@other

@c Set initial fonts (again)
@textfonts
@rm

@c Local variables:
@c eval: (add-hook 'write-file-hooks 'time-stamp)
@c page-delimiter: "^\\\\message"
@c time-stamp-start: "def\\\\palatinoversion{"
@c time-stamp-format: "%:y-%02m-%02d.%02H"
@c time-stamp-end: "}"
@c End:
